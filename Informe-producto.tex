\documentclass{article}
%%%%%%%%%%%%%%%%%%%%%%%%%%%%% Define Article %%%%%%%%%%%%%%%%%%%%%%%%%%%%%%%%%%
%%%%%%%%%%%%%%%%%%%%%%%%%%%%%%%%%%%%%%%%%%%%%%%%%%%%%%%%%%%%%%%%%%%%%%%%%%%%%%%

%%%%%%%%%%%%%%%%%%%%%%%%%%%%% Using Packages %%%%%%%%%%%%%%%%%%%%%%%%%%%%%%%%%%
\usepackage{geometry}
\usepackage{graphicx}
\usepackage{amssymb}
\usepackage{amsmath}
\usepackage{amsthm}
\usepackage{empheq}
\usepackage{mdframed}
\usepackage{booktabs}
\usepackage{lipsum}
\usepackage{graphicx}
\usepackage{color}
\usepackage{psfrag}
\usepackage{pgfplots}
\usepackage{bm}
%%%%%%%%%%%%%%%%%%%%%%%%%%%%%%%%%%%%%%%%%%%%%%%%%%%%%%%%%%%%%%%%%%%%%%%%%%%%%%%

% Other Settings

%%%%%%%%%%%%%%%%%%%%%%%%%% Page Setting %%%%%%%%%%%%%%%%%%%%%%%%%%%%%%%%%%%%%%%
\geometry{letterpaper, margin=2.54cm}

%%%%%%%%%%%%%%%%%%%%%%%%%% Define some useful colors %%%%%%%%%%%%%%%%%%%%%%%%%%
\definecolor{ocre}{RGB}{243,102,25}
\definecolor{mygray}{RGB}{243,243,244}
\definecolor{deepGreen}{RGB}{26,111,0}
\definecolor{shallowGreen}{RGB}{235,255,255}
\definecolor{deepBlue}{RGB}{61,124,222}
\definecolor{shallowBlue}{RGB}{235,249,255}
%%%%%%%%%%%%%%%%%%%%%%%%%%%%%%%%%%%%%%%%%%%%%%%%%%%%%%%%%%%%%%%%%%%%%%%%%%%%%%%

%%%%%%%%%%%%%%%%%%%%%%%%%% Define an orangebox command %%%%%%%%%%%%%%%%%%%%%%%%
\newcommand\orangebox[1]{\fcolorbox{ocre}{mygray}{\hspace{1em}#1\hspace{1em}}}
%%%%%%%%%%%%%%%%%%%%%%%%%%%%%%%%%%%%%%%%%%%%%%%%%%%%%%%%%%%%%%%%%%%%%%%%%%%%%%%

%%%%%%%%%%%%%%%%%%%%%%%%%%%% English Environments %%%%%%%%%%%%%%%%%%%%%%%%%%%%%
\newtheoremstyle{mytheoremstyle}{3pt}{3pt}{\normalfont}{0cm}{\rmfamily\bfseries}{}{1em}{{\color{black}\thmname{#1}~\thmnumber{#2}}\thmnote{\,--\,#3}}
\newtheoremstyle{myproblemstyle}{3pt}{3pt}{\normalfont}{0cm}{\rmfamily\bfseries}{}{1em}{{\color{black}\thmname{#1}~\thmnumber{#2}}\thmnote{\,--\,#3}}
\theoremstyle{mytheoremstyle}
\newmdtheoremenv[linewidth=1pt,backgroundcolor=shallowGreen,linecolor=deepGreen,leftmargin=0pt,innerleftmargin=20pt,innerrightmargin=20pt,]{theorem}{Theorem}[section]
\theoremstyle{mytheoremstyle}
\newmdtheoremenv[linewidth=1pt,backgroundcolor=shallowBlue,linecolor=deepBlue,leftmargin=0pt,innerleftmargin=20pt,innerrightmargin=20pt,]{definition}{Definition}[section]
\theoremstyle{myproblemstyle}
\newmdtheoremenv[linecolor=black,leftmargin=0pt,innerleftmargin=10pt,innerrightmargin=10pt,]{problem}{Problem}[section]
%%%%%%%%%%%%%%%%%%%%%%%%%%%%%%%%%%%%%%%%%%%%%%%%%%%%%%%%%%%%%%%%%%%%%%%%%%%%%%%

%%%%%%%%%%%%%%%%%%%%%%%%%%%%%%% Plotting Settings %%%%%%%%%%%%%%%%%%%%%%%%%%%%%
\usepgfplotslibrary{colorbrewer}
\pgfplotsset{width=8cm,compat=1.9}
%%%%%%%%%%%%%%%%%%%%%%%%%%%%%%%%%%%%%%%%%%%%%%%%%%%%%%%%%%%%%%%%%%%%%%%%%%%%%%%

%%%%%%%%%%%%%%%%%%%%%%%%%%%%%%% Title & Author %%%%%%%%%%%%%%%%%%%%%%%%%%%%%%%%
\author{Gustavo Vergara}
%%%%%%%%%%%%%%%%%%%%%%%%%%%%%%%%%%%%%%%%%%%%%%%%%%%%%%%%%%%%%%%%%%%%%%%%%%%%%%%


\begin{document}

\begin{titlepage}
    \centering
    
    
    \vspace{3cm}
    {\scshape\large DOCUMENTO CON ESPECIFICACIÓN DE REQUERIMIENTOS DEL SISTEMA DE GESTION DE CITAS MEDICAS PARA LA CLINICA REGIONAL DE MONTELIBANO \par}
    \vspace{7cm}
    \textbf\large\scshape{\par}
         \vspace{0.5cm}
         
    {\Large Vergara Pareja Gustavo\par}
    \vspace{7cm}
    {\scshape\Large Tecnologia en Analisis y Desarrollo de Software \par}
    \vspace{1cm}
    {\scshape\Large SENA - Centro Agropecuario Regional Cauca\par}
    \vspace{1cm}
    {\Large \today \par}
    \end{titlepage}

\tableofcontents

\newpage

\begin{flushleft}
    \large \textbf{EVIDENCIA A SOLUCIONAR}\\
    \vspace{1cm}
    
    \large Evidencia de producto: GA1-220501092-AA4-EV02 documento con especificación de requerimientos
    Respecto a lista de requerimientos el aprendiz deberá agregar una sección donde se describa cada requisito usando los siguientes elementos del estándar IEEE830.
    \begin{itemize}
    \item Perspectiva del producto.
    \item Funciones del producto.
    \item Características de los usuarios.
    \item Restricciones.
    \item Requisitos funcionales (formato de casos de uso).
    \item Requisitos no funcionales.
    \end{itemize}
    Respecto a la lista de requerimientos el aprendiz deberá agregar una sección donde se describa cada requisito usando la estructura de historias de usuario con los siguientes elementos por historia:
    \begin{itemize}
    \item Número de historia (priorizada).
    \item Nombre de la historia.
    \item Usuario.
    \item Puntos estimados de esfuerzo.
    \item Descripción de la historia de usuario.
    \item Observaciones.
    \item Criterios de aceptación.
    \end{itemize}
    \end{flushleft}
    \newpage


\section{Introducción}

Las Tecnologías de la Información y la Comunicación (TIC) han revolucionado la forma en que trabajamos y nos comunicamos en todos los ámbitos de la vida. En el área de ocupación, la incorporación de las TIC puede traer numerosas mejoras en términos de eficiencia, productividad y colaboración. En este informe, exploraremos las mejoras que ofrece la incorporación de las TIC en las herramientas ofimáticas y colaborativas, así como los aspectos procesales del área de ocupación que se pueden fortalecer con su implementación.

\section{Mejoras en las herramientas ofimáticas y colaborativas}

La incorporación de las TIC en las herramientas ofimáticas y colaborativas ofrece numerosas mejoras en los procesos de trabajo. Algunas de estas mejoras incluyen:

\begin{enumerate}
  \item Mayor eficiencia en la gestión de documentos: Las herramientas ofimáticas basadas en la nube permiten un acceso fácil y rápido a los documentos desde cualquier lugar y en cualquier momento. Esto facilita la colaboración en tiempo real y evita la pérdida de información.
  
  \item Mejor comunicación y colaboración: Las herramientas de comunicación en línea, como el correo electrónico, las videoconferencias y las plataformas de mensajería instantánea, permiten una comunicación más rápida y efectiva entre los miembros del equipo. Además, las herramientas colaborativas, como los documentos compartidos y los espacios de trabajo virtuales, facilitan la colaboración en proyectos y tareas.
  
  \item Automatización de tareas: Las TIC permiten la automatización de tareas repetitivas y tediosas, lo que ahorra tiempo y reduce errores. Por ejemplo, el uso de hojas de cálculo y software de contabilidad automatiza los cálculos y la generación de informes financieros.
\end{enumerate}

\section{Aspectos procesales fortalecidos con la incorporación de las TIC}

La incorporación de las TIC en el área de ocupación fortalece varios aspectos procesales. Algunos de estos aspectos incluyen:

\begin{enumerate}
  \item Gestión de proyectos: Las herramientas de gestión de proyectos basadas en la nube permiten una planificación y seguimiento más eficientes de los proyectos. Estas herramientas facilitan la asignación de tareas, el seguimiento del progreso y la colaboración entre los miembros del equipo.
  
  \item Gestión del tiempo: Las aplicaciones y herramientas de gestión del tiempo ayudan a los empleados a organizar y priorizar sus tareas, lo que mejora la productividad y reduce el estrés. Estas herramientas también permiten el seguimiento del tiempo dedicado a cada tarea, lo que facilita la facturación y la evaluación del rendimiento.
  
  \item Gestión del conocimiento: Las TIC permiten la creación y gestión de bases de conocimiento en línea, donde los empleados pueden acceder a información relevante y compartir su conocimiento con otros. Esto facilita el aprendizaje y la colaboración dentro de la organización.
\end{enumerate}

\section{Presentación creativa del informe}

Para presentar el informe de forma creativa, se puede utilizar una herramienta TIC en línea, como una presentación de diapositivas interactiva, un video animado o una infografía. Estas herramientas permiten transmitir la información de manera visualmente atractiva y fácil de entender. Además, se pueden utilizar elementos como cuentos, caricaturas o personajes animados para hacer la presentación más entretenida y memorable.

\section{Conclusión}

La incorporación de las TIC en el área de ocupación ofrece numerosas mejoras en términos de eficiencia, productividad y colaboración. Las herramientas ofimáticas y colaborativas basadas en las TIC permiten una gestión más eficiente de documentos, una comunicación y colaboración más efectivas, y la automatización de tareas repetitivas. Además, las TIC fortalecen aspectos procesales como la gestión de proyectos, la gestión del tiempo y la gestión del conocimiento. Al presentar el informe de forma creativa utilizando herramientas TIC en línea, se puede transmitir la información de manera visualmente atractiva y memorable.

\end{document}
